\chapter{Introducción específica} % Main chapter title

\label{Chapter2}

%----------------------------------------------------------------------------------------
%	SECTION 1
%----------------------------------------------------------------------------------------
En este capítulo se enumeran los requisitos que la solución debe cumplir y luego se describen principalmente las herramientas utilizadas durante el desarrollo para el entrenamiento y despliegue de los modelos para satisfacerlos.

\section{Requerimientos}

En esta sección se detallan la funcionalidad que el trabajo debía cubrir y las restricciones de implementación.


\begin{enumerate}
	\item Requerimientos funcionales
	\begin{enumerate}
		\item El sistema debe poder detectar la categoría de un reclamo escrito en lenguaje natural.
		\item El sistema debe poder detectar la categoría de una consulta escrita en lenguaje natural.
		\item El usuario debe poder utilizar los resultados de la clasificación desde una base de datos.
		\item El proceso debe ser capaz de interpretar errores de ortografía.
		\item El proceso debe ser capaz de adaptarse a distinta cantidad de palabras en el mensaje.
		\item La solución debe ejecutarse en forma \textit{batch}, corriendo diariamente y tomando los casos del día anterior.
	\end{enumerate}
	\item Requerimientos no funcionales
	\begin{enumerate}
		\item El sistema debe estar desarrollado en lenguaje Python.
		\item El código debe ser versionado con Git.
		\item La solución debe estar desplegada sobre infraestructura de Google Cloud Platform.
		\item La salida de los modelos debe ser almacenada en BigQuery.
		\item El proceso debe ser ejecutado a través del orquestador Apache Airflow.
	\end{enumerate}
	\item Requerimientos de testing
	\begin{enumerate}
		\item Se deben generar métricas de desempeño de los modelos con el dataset de entrenamiento y de prueba.
	\end{enumerate}
	\item Requerimientos de documentación
	\begin{enumerate}
		\item Se debe confeccionar un documento con el diseño de la arquitectura de alto nivel.
		\item Se debe confeccionar un documento con el diseño de los modelos de IA.
		\item Se debe confeccionar un documento que especifique los datos que consumen los modelos y su origen.
	\end{enumerate}
\end{enumerate}



\section{Preprocesamiento del texto}

En esta sección se introducen las herramientas utilizadas para realizar el limpieza y preparación del texto que sirvió tanto para la parte de entrenamiento como para la parte de predicción en el despliegue.

\subsection{NLTK}



\section{Modelos de inteligencia artificial utilizados}



\section{Herramientas de software utilizadas}