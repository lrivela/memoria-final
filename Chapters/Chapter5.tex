% Chapter Template

\chapter{Conclusiones} % Main chapter title

\label{Chapter5} % Change X to a consecutive number; for referencing this chapter elsewhere, use \ref{ChapterX}

En este capítulo se presentan los resultados del trabajo y posibles mejoras a futuro.

\section{Conclusiones generales}

Se desarrolló un sistema de clasificación de reclamos de manera exitosa, incluyendo su despliegue en un ambiente de desarrollo, cumpliendo con todos los requerimientos funcionales y no funcionales. Se pudo cumplir con la planificación original y finalizar el total de las tareas, a pesar de que las actividades relacionadas al entrenamiento de los modelos llevaron más horas de las asignadas inicialmente.

A continuación se listan los logros del trabajo final:
\begin{itemize}
	\item Identificar la categoría principal de los reclamos con un F1 de 0,8924.
	\item Identificar la subcategoría de los reclamos con un F1 superior a 0,77 excepto para dos clasificadores.
	\item Haber desplegado un proceso de predicción en un ambiente de desarrollo, utilizando la infraestructura y herramientas que el área dispuso.
\end{itemize}

A continuación, se destacan las materias de la carrera de mayor relevancia para la realización de este trabajo:
\begin{itemize}
	\item Procesamiento del lenguaje natural: fue extremadamente importante para aplicar técnicas de preprocesamiento de texto, vectorización y clasificación de texto.
	\item Bases de datos: para poder realizar la extracción de los datos y para almacenar los resultados.
	\item Gestión de proyectos: para elaborar un plan y bajarlo a un calendario específico, que posibilitó cumplir con los plazos estipulados.
\end{itemize}


\section{Próximos pasos}

Luego de haber finalizado este trabajo, se identifican dos tipos de mejoras posibles: 
\begin{itemize}
	\item De desempeño.
	\item De funcionalidad.
\end{itemize}

Una alternativa para mejorar el desempeño de los clasificadores puede ser probar modelos de IA más avanzados y con mayor cantidad de parámetros, como la versión \textit{large} de BERT o incluso probando ROBERTA, que requieren un poder mayor de cómputo que el utilizado. Esto puede resultar particularmente importante para el clasificador L2\_E cuya métrica F1 resultó inferior a 0,7, y también para el clasificador L1 por su impacto en la clasificación final de las subcategorías.

En cuanto a la funcionalidad, durante el transcurso de este trabajo, la empresa ha lanzado nuevos productos que involucran nuevas categorías de reclamos. Cuando se tenga un dataset de tamaño significativo, se puede re-entrenar el clasificador L1 para incluir las nuevas categorías y los clasificadores L2 para las subcategorías correspondientes.

Por otro lado, se pueden entrenar nuevos clasificadores para las categorías L3 más importantes, y de esta forma, cubrir el árbol de clasificación completo del proceso de atención de reclamos y consultas de la empresa.
