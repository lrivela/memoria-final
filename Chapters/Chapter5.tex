% Chapter Template

\chapter{Conclusiones} % Main chapter title

\label{Chapter5} % Change X to a consecutive number; for referencing this chapter elsewhere, use \ref{ChapterX}

\section{Conclusiones generales}


Se destacan las siguientes materias de la carrera para la realización de este trabajo:
\begin{itemize}
	\item Procesamiento del lenguaje natural: fue extremadamente importante para aplicar técnicas de preprocesamiento de texto, vectorización y clasificación de texto.
	\item Bases de datos: para poder realizar la extracción de los datos y para almacenar los resultados.
	\item Gestión de proyectos: para elaborar un plan y bajarlo a un calendario específico, que posibilitó cumplir con los plazos estipulados.
\end{itemize}


\section{Próximos pasos}

Luego de haber finalizado este trabajo, se identifican dos tipos de mejoras posibles: 
\begin{itemize}
	\item De desempeño.
	\item De funcionalidad.
\end{itemize}

Una alternativa para probar mejorar el desempeño de los clasificadores puede ser probar modelos de IA más avanzados y con mayor cantidad de parámetros, como la versión \textit{large} de BERT o incluso probando ROBERTA, que requieren un poder mayor de cómputo que los utilizados. Esto puede resultar particularmente importante para el clasificador L2\_E cuya métrica F1 resultó inferior a 0,7, y también para el clasificador L1 por su impacto en la clasificación final de las subcategorías.

En cuanto a la funcionalidad, durante el transcurso de este trabajo, la empresa ha lanzado nuevos productos que involucran nuevas categorías de reclamos. Cuando se tenga un dataset significativo, se puede re-entrenar el clasificador L1 para incluir las nuevas categorías y los clasificadores L2 para las subcategorías correspondientes.

Por otro lado, se pueden entrenar nuevos clasificadores para las categorías L3, y de esta forma, cubrir el árbol de clasificación completo del proceso de atención de reclamos y consultas de la empresa.
